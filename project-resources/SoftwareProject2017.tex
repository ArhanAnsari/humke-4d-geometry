%Last Edit 1/17/2017
\documentclass{article}

\usepackage{epsfig,amsmath,amssymb, enumerate}  

\pagestyle{plain}
\setlength{\textwidth}{5.5in}
\setlength{\textheight}{8.2in}
\setlength{\topmargin}{-.4 in}
\setlength{\oddsidemargin}{.5in} 
\setlength{\parindent}{0pt}
\setlength{\parskip}{12pt}

\def\reals{\mathbb{R}}

\begin{document}
\begin{center}
{\huge Software Project Outline -- 2017}
\end{center}

\bigskip
\begin{enumerate}[]
\item \begin{Large}{Some Notation}\end{Large}

I'll use $x_i$, $i=1,2,3,4$ for the Euclidean coordinates and $r,s,t,$ for parametric variables.
\item \begin{Large}Basic Functionality\end{Large}
\begin{enumerate}[1.]
\item A dimension is selected (either 2,3, or 4)
\item Objects can be assumed to be in a $-10$ to $10$ bounding box, i.e.$-10\le x_i\le 10$ for all $i$.
\end{enumerate}
\item \begin{Large}Inputing Objects\end{Large}
\begin{enumerate}[1.]
\item As a Cartesian Equation
\[
x_1^2+3x+2^2+4x_4^2\le 1.
\]
\item As a Convex Hull of Points
\begin{enumerate}[a.]
\item Points are entered (here in 3D where at lease 4 should be required)
\[
p_1=(1,0,0),\ p_2=(0,1,0),\ p_3=(0,0,1),\ p_4=(1,1,1).
\]
\item $\text{HULL}(p_1,p_2,p_3,p_4)=\{s_1\cdot(1,0,0)+s_2\cdot(0,1,0)+s_3\cdot(0,0,1)+s_4\cdot(1,1,1)\}$.
\end{enumerate}
\item Parametrically
\begin{align*}
x_1&=r\cos(4\pi t)-s\sin(4\pi t) \\
x_2&=r\sin(4\pi t)+s\cos(4\pi t) \\
x_3&=5t
\end{align*}
with these equations and the limits below entered by the user.
\[
0\le r\le 1,\quad 0\le s\le 1,\quad -1\le t\le 1.
\]

\end{enumerate}
\item \begin{Large}Viewing Sections\end{Large} (3D example)
\begin{enumerate}[1.]
\item The user specifies a coordinate direction (either $x_1,\ x_2$ or $x_3$ and a color.
\item Sections orthogonal to the specified direction are shown via ``slider'' or ``movie'' 
(user choice) beginning with -10 and ending with 10.
\item Sections of 4D objects are 3D solids so picturing a particular section and pulling it around as per {\it Mathematica} would be nice but is not necessary.
\end{enumerate}
\item \begin{Large}Viewing Objects\end{Large}
\begin{enumerate}[1.]
\item In 2D this is clear.
\item In 3D this is "pretty clear" but again the {\it Mathematica} style viewing would be nice.
\item In 4D this may only be meaningful in the convex hull case. Then a choice needs to be made as to whether ``perspective view" or ``projection view" is seen. Perhaps these could be toggled.
\end{enumerate}

\end{enumerate}

\end{document}
